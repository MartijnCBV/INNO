\documentclass{article}

\usepackage{float}
\usepackage{graphicx}
\usepackage[dutch]{babel}

\usepackage{hyperref}
\hypersetup{
    colorlinks=true,
    linkcolor=black,
    filecolor=blue,      
    urlcolor=blue,
    pdftitle={FO Digtal Twins API},
    pdfpagemode=FullScreen,
}

\title{Kennisgaten}
\author{Martijn Voorwinden 1776622}
\date{10-11-2022}

\newcommand{\ls}[8]{
    \maketitle
    \section*{Kennisgaten}
    \subsection*{Inleiding}
    #1
    \subsection*{Criteria}
    \begin{itemize}
        \item Kennis opdoen en kunnen toepassen binnen context.
    \end{itemize}
    \subsection*{Learning / Research Story}
    \textbf{#2}
    % ss van story in backlog
    #3
    \subsubsection*{Acceptatiecriteria}
    % itemize
    #4
    \subsubsection*{Definition of Done (DoD)}
    % itemize
    #5
    \subsubsection*{Estimate}
    #6
    \subsubsection*{Artifacten / Realisatie}
    #7
    \subsubsection*{Bronnen}
    #8
}

\begin{document}
    \ls
    {
        Voor provincie Utrecht moet er een datacatalogus gemaakt worden om de me-
        dewerkers makkelijker toegang te geven tot de verschillende databronnen in de
        verschillende afdelingen van de provincie. Dit is nodig om te zorgen dat er geen
        dubbele datasets worden aangeschaft en om de zoektijd naar relevante datasets
        te verkleinen. Mijn rol binnen dit project is developer.
    }
    {
        Als student wil ik leren hoe ik een graphical user interface (GUI) doormiddel van HTML en CSS maak in elm zodat ik een GUI kan maken voor het project.
    }
    {
        \begin{figure}[H]
            \includegraphics[width=\textwidth,height=\textheight,keepaspectratio]{ls_01.png}
            \caption{\href{https://dev.azure.com/HU-HBO-ICT/2022-INNO-400-Provincie\%20Utrecht/_backlogs/backlog/2022-INNO-400-Provincie\%20Utrecht\%20Team/Epics/?workitem=96710}{De learning story in devops.}}
            \label{fig:ls_01}
        \end{figure}
        Voor het project wil ik graag weten hoe ik een GUI kan maken in elm doormiddel van HTML en CSS,\
        zodat gebruikers gebruik kunnen maken van het gemaakte product.
    }
    {
        \begin{itemize}
            \item Er zijn verschillende bronnen over een GUI maken in elm gevonden.
            \item Er is een test GUI gemaakt.
        \end{itemize}
    }
    {
        \begin{itemize}
            \item De learning story is behaald wanneer deze op niveau is.
            \item De learning is gereviewd door een medestudent.
            \item De docent gaat akkoord met het resultaat van de learning story.
        \end{itemize}
        %\begin{itemize}
        %    \item Er zijn bronnen gevonden.
        %    \item De gevonden bronnen zijn bestudeerd.
        %    \item De geleerde kennis is toegepast in een test applicatie.
        %\end{itemize}
    }
    {
        De estimate voor deze learning story is 1-2 uren, dit omdat het een relatief simpel iets is.
    }
    {
        \begin{itemize}
            \item \url{https://github.com/MartijnCBV/INNO/blob/ls_01_elm_expose_gui/frontend/src/Page/Home.elm}
        \end{itemize}
    }
    {
        \begin{itemize}
            \item \url{https://guide.elm-lang.org/architecture/buttons.html}
            \item \url{https://guide.elm-lang.org/architecture/text_fields.html}
            \item \url{https://guide.elm-lang.org/architecture/forms.html}
            \item \url{https://github.com/rtfeldman/elm-spa-example}
        \end{itemize}
    }
\end{document}