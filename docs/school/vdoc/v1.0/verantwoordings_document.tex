\documentclass[12pt, a4paper]{report}
\usepackage[dutch]{babel}
\usepackage[utf8]{inputenc}
\usepackage{listings}
\usepackage{csquotes}
\usepackage{vhistory}

\usepackage[backend=biber,style=numeric,sorting=none]{biblatex}
\addbibresource{refs.bib}

\usepackage{graphicx}
\graphicspath{ {./images/} }

\usepackage[explicit]{titlesec}
\titleformat{\chapter}[display]{\large\bfseries}{}{0pt}{\large\MakeUppercase{#1}}
\titlespacing*{\chapter}{0pt}{0pt}{20pt}

\tolerance=1
\hyphenpenalty=10000
\hbadness=10000

\title{INNO Verantwoordings document}
\author{Martijn Voorwinden  1776622}
\date{2022-11-01}

\begin{document}
    \maketitle
    \newpage
    \tableofcontents
    \chapter{Opdracht}
    Voor de provincie Utrecht moet er een proof of concept (PoC) van een data catalogus ontwikkelt worden,
    dit is nodig om de verschillende databronnen binnen de provincie makkelijker vindbaar te maken.

    \chapter{Gebruikte technologiën}
    Er is besloten om de data catalogus te maken in microsoft purview en een web front-end in elm.
    \section{Microsoft purview}
    Microsoft purview is een data management systeem wat onder andere gebruikt kan worden om een data catalogus op te zetten en in verscheidene databronnen te zoeken \cite{purview}.
    Er is voor microsoft purview gekozen omdat de Provincie Utrecht al met veel microsoft producten werkten en dus het koppelen met andere interne \\ producten makkelijker is.
    \section{Elm}
    Elm is een functionele programeertaal die compileert naar JavaScript en als een alternatief voor JavaScript gebruikt kan worden \cite{elm}.
    Er is voor elm gekozen omdat het sterke getypeerd is en een rigide compiler heeft,
    dit zorgt ervoor dat er geen runtime errors zijn, maar alle errors op compile time worden gegooid.
    Elm is ook een technologie die het team graag wil leren.

    \chapter{Code structuur}
    \section{Opdeling in modules (Elm)}
    Om de code overzichtelijk en onderhoudbaar te houden moet het opgedeeld worden in verschillende modules,
    de waarde van deze modules kan beoordeeld worden aan de hand van de officiele elm richtlijnen \cite{elm-modules}.
    \section{Code stijl (Elm)}
    Om de code makkelijk overleverbaar te maken moet er gebruik gemaakt worden van een gestandaardiseerde stijl,
    daarom wordt gebruik gemaakt van de geadviseerde stijl uit de officiële elm documentatie \cite{elm-style}.

    \newpage
    \nocite{*}
    \printbibliography
\end{document}